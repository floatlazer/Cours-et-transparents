\documentclass{beamer}
\usepackage{tikz}
\usetikzlibrary{patterns}
\usetikzlibrary{shadows}
%\usetikzlibrary{trees}
\usepgflibrary{shapes.arrows}
\usetikzlibrary{arrows}
\usetikzlibrary{shapes}
\usetikzlibrary{decorations.pathmorphing}
%\usepackage{tkz-euclide}
\usepackage{amsmath, amssymb}
\usepackage[utf8x]{inputenc}
\usepackage{colortbl}
\usepackage[french]{babel}
\usepackage{listings}
\usepackage{times}
\usepackage{xcolor}
\usepackage{fp}
\usepackage{algorithmic}
%\makeatletter

\newcommand{\R}{\mathbb{R}}
%\newcommand{\C}{\mathbb{C}}

\definecolor{royalblue}{rgb}{0.3,0.5,0.9}
\definecolor{lightblue}{rgb}{0.7,0.7,1.0}
\definecolor{navyblue}{rgb}{0.1,0.2,0.7}
\definecolor{Maroon}{rgb}{0.5,0.5,0.1}

\title{Programming GPGPU}
\author[Juvigny]{Xavier Juvigny}
\institute{ONERA/CHP}
\date[C2013]{March 2013}
\subject{GPGPU}

%\pgfdeclareimage[interpolate=true,width=2cm,height=0.5cm]{logoEsilv}{esilv}
\pgfdeclareimage[interpolate=true,width=.45\textwidth]{cudavisu1}{CudaVisual1}
\pgfdeclareimage[interpolate=true,width=.45\textwidth]{cudavisu2}{CudaVisual2}

\lstloadlanguages{[ISO]C++,Python,[90]Fortran}
\lstset{
    language=C++
  , frame=single
  , basicstyle=\scriptsize
  , keywordstyle=\color{blue}\bfseries
  , commentstyle=\color{red}
  , stringstyle=\color{magenta}\ttfamily
  , backgroundcolor=\color{verylightgray}
}

%\usetheme{Onera2007}
\setbeamercovered{invisible} 
\setbeamertemplate{navigation symbols}{}

\definecolor{green!50!black}{rgb}{0.3,0.5,0.9}
\definecolor{navyblue}{rgb}{0.1,0.2,0.7}
\definecolor{darkgreen}{rgb}{0. 0.2 0}
\definecolor{darkred}{rgb}{0.4 0 0}
\definecolor{green!50!black}{rgb}{0.2 0. 0.2}
\definecolor{lightgreen!50!black}{rgb}{0.4 0. 0.4}
\definecolor{midblue}{rgb}{0. 0.2 0.4}
\definecolor{verylightgray}{rgb}{0.95 0.95 1.0}
\definecolor{lightyellow}{rgb}{1.0 0.95 0.75}

% Compteurs pour certaines images
\newcounter{gp}
\newcounter{lp}


\lstloadlanguages{[ISO]C++,Python,[90]Fortran}
\lstset{
    language=C++
  , frame=single
  , basicstyle=\scriptsize
  , keywordstyle=\color{blue}\bfseries
  , commentstyle=\color{red}
  , stringstyle=\color{magenta}\ttfamily
  , backgroundcolor=\color{verylightgray}
}

\begin{document}
\frame[plain]{\maketitle}

\section{GPGPU architecture nVIDIA/ATI}

\subsection{CPU--GPU overview}

\begin{frame}{CPU--GPU overview}

\begin{itemize}
\item The CPU uses the GPGPU as a computing coprocessor for some computations
  as a SIMD computing device for some well adapted algorithms.
\item CPU and GPGPU are both multi--cores devices and have hierarchical memory
architecture.
\end{itemize}

\begin{center}
\begin{tikzpicture}
\draw (-4,-2) rectangle (-1,2) (1,-2) rectangle (5,2);
\draw[fill=cyan] (-3.75,-0.5) rectangle (-1.25,1.75)
                 ( 1.25,-0.5) rectangle ( 4.75,1.75) ;
\node[color=black] at (-2.5,1.50){\scriptsize CPU multi--c{\oe}ur};
\draw[fill=orange] (-3.6,0.25) rectangle (-1.4,1.25)
                   (1.4,-0.25) rectangle (4.6,1.25);
\draw[fill=yellow] (-3.3,0.35) rectangle (-2.5,0.70)
                   (-3.3,0.80) rectangle (-2.5,1.15)
                   (-2.3,0.35) rectangle (-1.5,0.70)
                   (-2.3,0.80) rectangle (-1.5,1.15);
\node[color=black] at (-2.9,0.50){\scriptsize core 1};
\node[color=black] at (-2.9,0.97){\scriptsize core 2};
\node[color=black] at (-1.9,0.50){\scriptsize core 4};
\node[color=black] at (-1.9,0.97){\scriptsize core 3};
\draw[fill=red] (-3.6,-0.25) rectangle (-1.4,0.15);
\node[color=black] (ci) at (-2.5,-0.05){\scriptsize Internal cache};
\draw[fill=red] (-3.6,-0.65) rectangle (-1.4,-1.15);
\node[color=black] (ce) at (-2.5,-0.9){\scriptsize External cache};
\draw[fill=green] (-3.8,-1.4) rectangle (-1.2,-1.95);
\node[color=black] (me) at (-2.5,-1.65){\scriptsize RAM for CPU};
\draw[thick,blue,<->] (ci)--(ce);
\draw[thick,blue,<->] (ce)--(me);
\node[double arrow, draw, left color=blue, right color=royalblue]  at (0,-1.5){\scriptsize PCI-Express};
\node[color=black]  at (+3.0,1.50){\scriptsize GPU $n$ multi-processors};
\foreach \c in {0,0.1,0.2,0.3} {
\draw[fill=yellow]  (4.5-\c,1.1-\c) rectangle (1.8-\c,0.2-\c);
  \foreach \d in {0,0.15,0.3,...,0.9} {
    \draw[fill=Maroon]  (2.0+\d-\c,1.-\c) rectangle (2.1+\d-\c,0.7-\c);
  }
}
\node at (3.35,0.55){\scriptsize SIMD units};
\node (ml) [draw, left color=red, right color=red] at (3.0,0.1){\tiny Local memory};
\node (mgpu) [draw, left color=red, right color=red] at (3.,-1.5){\scriptsize
  RAM for GPU};
\draw[thick,blue,<->] (ml)--(mgpu);
\end{tikzpicture}
\end{center}

\end{frame}

\subsection{Close view of GPGPU architecture}

\begin{frame}{Close view of GPGPU architecture}

\begin{minipage}{55mm}
\begin{itemize}
\item GPGPU : \textcolor{orange}{Set of $N$ tiny SIMD independant devices sharing same
    global memory} : $N$ multiprocessors;
\item Multiprocessor : \textcolor{orange}{Small SIMD device with :} {\scriptsize
  \begin{itemize}
    \item $k$ synchronized ALU;
    \item 1 instruction decoder;
    \item Three shared memory for all ALUs (included 2 cache memories)
    \item R  registers distributed among the AlUs
      (locals for each \textsl{thread}) (Example FERMI : 16384)
  \end{itemize}
}
\end{itemize}
\end{minipage}
\begin{minipage}{5cm}
\begin{center}
\begin{tikzpicture}
\draw[fill=cyan]   (-2cm,6cm) rectangle (2cm,1.5cm);
\draw[fill=red] (-2cm,1cm) rectangle (2cm,0.1cm);
\node at (-1.35,5.8){\tiny Device};
\node at (-0.75,0.8){\tiny Global(Device)  Memory};
\draw[fill=orange] (-1.5cm,5.5cm) rectangle (1.8cm,2.5cm);
\node at (-0.55,5.3){\tiny Multiprocessor N};
\draw[dashed,thick] (-0.55,5.2) -- (-0.75,4.6);
\draw[fill=orange] (-1.7cm,4.75cm) rectangle (1.6cm,1.75cm);
\node at (-0.75,4.5){\tiny Multiprocessor 2};
\draw[fill=orange] (-1.9cm,4.55cm) rectangle (1.4cm,1.55cm);
\node at (-0.95,4.3){\tiny Multiprocessor 1};
\draw[fill=yellow] (-1.7cm,4.1cm) rectangle (1.2cm,3.8cm);
\node at (-0.25,3.95){\tiny Local memory};
\node[draw,left color=yellow, right color=yellow] at (-1.5cm,3.5cm){\tiny r};
\node[draw,left color=yellow, right color=yellow] at (-1.05cm,3.5cm){\tiny r};
\node[draw,left color=yellow, right color=yellow] at (-0.35cm,3.5cm){\tiny r};
\draw[fill=lightyellow] (-1.6cm,3.2cm) rectangle (0.3cm,3cm);
\node (p1)[draw,left color=green, right color=cyan] at (-1.45cm,3.1cm){\tiny $p_{1}$};
\node (p2)[draw,left color=green, right color=cyan] at (-1cm,3.1){\tiny $p_{2}$};
\draw[dotted,thick] (-0.75,3.1) -- (-0.25,3.1);
\node (pk)[draw,left color=green, right color=cyan] at (-0.25cm,3.1cm){\tiny
  $p_{k}$};
\node[draw,left color=green, right color=cyan] at (0.6,3.1)
{\tiny \begin{minipage}{0.7cm}Instr.\\ Unit\end{minipage}};
\draw[fill=yellow] (-1.7cm,2.7cm) rectangle (1.1cm,2.2cm);
\node at (-0.3,2.4){\tiny Texture cache};
\draw[fill=yellow] (-1.7cm,2.1cm) rectangle (1.1cm,1.6cm);
\node at (-0.3cm,1.8){\tiny Constant cache};
\draw[<->] (-1.45cm,1cm) -- (p1);
\draw[<->] (-1.00cm,1cm) -- (p2);
\draw[<->] (-0.25cm,1cm) -- (pk);
\draw[->]  (0.9cm,1cm) -- (0.9cm,1.6cm);
\draw[->]  (0.7cm,1cm) -- (0.7cm,2.2cm);
\draw[<->] (-1.3cm,3.3cm) -- (-1.3cm,3.8cm);
\draw[<->] (-0.85cm,3.3cm) -- (-0.85cm,3.8cm);
\draw[<->] (-0.1cm,3.3cm) -- (-0.1cm,3.8cm);
\draw[->]  (-1.6cm,2.7cm) -- (-1.6cm,2.9cm);
\draw[->]  (-1.15cm,2.7cm) -- (-1.15cm,2.9cm);
\draw[->]  (-0.4,2.7cm) -- (-0.4,2.9cm);
\draw[->]  (-1.3cm,2.1cm) -- (-1.3cm,2.9cm);
\draw[->]  (-0.85cm,2.1cm) -- (-0.85cm,2.9cm);
\draw[->]  (-0.1cm,2.1cm) -- (-0.1cm,2.9cm);

\node[draw,left color=blue, right color=cyan] at (0.,-1.3cm){\large CPU + RAM};
\node[double arrow, rotate=90,draw, left color=blue, right color=royalblue]  at (0,-0.5){\tiny PCI-E};

\end{tikzpicture}
\end{center}
\end{minipage}
\end{frame}

\section{Execution model}

\begin{frame}[containsverbatim]{Extended C}

  \begin{itemize}
  \item \textbf{\textcolor{blue}{New déclarations}} : \textcolor{orange}{\small global, local, local, constant,...}
    \begin{lstlisting}
      global float filter[N];
      kernel void  convolve(float* image) {
      local float region[M];
      ...
    \end{lstlisting}
  \item \textbf{\textcolor{blue}{New reserved words}} : 
    \textcolor{orange}{\small get\_local\_id, get\_global\_id, \ldots}
    \begin{lstlisting}
      region[get_global_id(0)] = image[i];
    \end{lstlisting}
  \item \textbf{\textcolor{blue}{Intrinsics}} : 
    \textcolor{orange}{\small barrier}
    \begin{lstlisting}
      barrier(CLK_LOCAL_MEM_FENCE);
      image[j] = result;  }
  \end{lstlisting}
  \item \textbf{\textcolor{blue}{Execution API}} :
    \textcolor{orange}{\small Memory, symbol, execution management}
    \begin{lstlisting}
// Allocate memory on GPU
cl::Buffer Ad(context, CL_MEM_READ_WRITE, bytes, myImg);
    \end{lstlisting}
  \item \textcolor{blue}{\bf Function calling}
    \begin{lstlisting}
// 1000 threads per groups of 10 threads
q.enqueueNDRangeKernel(kernel, cl::NullRange, 
                       cl::NDRange(1000), cl::NDRange(10));
    \end{lstlisting}
  \end{itemize}
\end{frame}

\begin{frame}[containsverbatim]{Threads distributions : groups and local threads}

\begin{minipage}[c]{52mm}
\begin{itemize}
\item A kernel is executed as a grid of groups of threads
  \begin{itemize}
  \item 
    \textcolor{orange}{\scriptsize All threads share same data memory space}
  \end{itemize}
\item A group of threads is a set of threads which can work together with :
  \begin{itemize}
  \item \textcolor{orange}{\scriptsize synchronizing between themself}
  \item \textcolor{orange}{\scriptsize Sharing data through fast shared memory}
  \end{itemize}
\item Two threads being in different groups can't work together.
  \begin{itemize}
  \item 
  \textcolor{orange}{\scriptsize Atomic operations added from hardware version HW 1.1}
\end{itemize}
\end{itemize}
\end{minipage}
\begin{minipage}[c]{53mm}
\begin{tikzpicture}{56mm}
\draw[fill=cyan] (0,0) rectangle (1cm,7cm);
\node[color=black] at (2.5mm,68mm){\tiny \textbf{Host}};
\draw[fill=green] (6mm,50mm) node[draw,fill] (K1) {\tiny
  \begin{minipage}{6mm}Kernel\\ 1\end{minipage}};
\draw[fill=green] (6mm,30mm) node[draw,fill] (K2) {\tiny
  \begin{minipage}{6mm}Kernel\\ 2\end{minipage}};
\draw[color=white,->,thick] (1mm,60mm) -- (1mm,1mm);

\draw[fill=cyan] (15mm,0mm) rectangle (5cm,7cm);
\node at (20mm,68mm){\tiny \textbf{device}};
\draw[fill=green] (17mm,41mm) rectangle (48mm,65mm);
\node at (25mm,62mm){\tiny \textbf{Grid 1}};
\draw[fill=yellow] (22mm,55mm) node[draw,fill] (B00) {\tiny
  \begin{minipage}{6mm}Group\\(0,0)\end{minipage}};
\draw[fill=yellow] (32mm,55mm) node[draw,fill] (B10) {\tiny
  \begin{minipage}{6mm}Group\\(1,0)\end{minipage}};
\draw[fill=yellow] (42mm,55mm) node[draw,fill] (B20) {\tiny
  \begin{minipage}{6mm}GRoup\\(2,0)\end{minipage}};

\draw[fill=yellow] (22mm,46mm) node[draw,fill] (B01) {\tiny
  \begin{minipage}{6mm}Group\\(0,1)\end{minipage}};
\draw[fill=yellow] (32mm,46mm) node[draw,fill] (B11) {\tiny
  \begin{minipage}{6mm}Group\\(1,1)\end{minipage}};
\draw[fill=yellow] (42mm,46mm) node[draw,fill] (B21) {\tiny
  \begin{minipage}{6mm}Group\\(2,1)\end{minipage}};
\draw[thick, ->] (K1) -- (17mm,50mm);


\draw[fill=green] (21mm,11mm) rectangle (44mm,40mm);
\node at (29mm,38mm){\tiny \textbf{Grid 2}};

\draw[fill=yellow] (26mm,30mm) node[draw,fill] {\tiny
  \begin{minipage}{6mm}Group\\(0,0)\end{minipage}};
\draw[fill=yellow] (36mm,30mm) node[draw,fill] {\tiny
  \begin{minipage}{6mm}Group\\(1,0)\end{minipage}};
\draw[thick,->] (K2) -- (21mm,30mm);

\draw[thick,dashed] (27.7mm,42.5mm) -- (6.8mm,2.5mm);
\draw[thick,dashed] (36mm  ,42.5mm) -- (56mm,2.5mm);
\draw[thick,dashed] (27.7mm,49.5mm) -- (6.8mm,32mm);
\draw[thick,dashed] (36mm  ,49.5mm) -- (56mm,32mm);



\draw[fill=yellow, fill opacity=0.7] 
  (6.8mm,2.5mm) rectangle (56mm,32mm);
\node at (14mm,29mm) {\tiny Group (1,1)};
\matrix[fill=red,matrix anchor=south west,nodes=draw] at (13mm,3mm) {
\node{\begin{minipage}{6mm}\tiny Thread (0,0)\end{minipage}};  &&
\node{\begin{minipage}{6mm}\tiny Thread (1,0)\end{minipage}};  && 
\node{\begin{minipage}{6mm}\tiny Thread (2,0)\end{minipage}};  &&
\node{\begin{minipage}{6mm}\tiny Thread (3,0)\end{minipage}};  \\
\node{\begin{minipage}{6mm}\tiny Thread (0,1)\end{minipage}};  && 
\node{\begin{minipage}{6mm}\tiny Thread (1,1)\end{minipage}};  &&
\node{\begin{minipage}{6mm}\tiny Thread (2,1)\end{minipage}};  &&
\node{\begin{minipage}{6mm}\tiny Thread (3,1)\end{minipage}};  \\
\node{\begin{minipage}{6mm}\tiny Thread (0,2)\end{minipage}};  && 
\node{\begin{minipage}{6mm}\tiny Thread (1,2)\end{minipage}};  &&
\node{\begin{minipage}{6mm}\tiny Thread (2,2)\end{minipage}};  &&
\node{\begin{minipage}{6mm}\tiny Thread (3,2)\end{minipage}};  \\ 
};

%\draw[step=0.1cm,color=gray] (0,0) grid (5cm,7cm);
%\draw[step=0.5cm] (0,0) grid (5cm,7cm);

\end{tikzpicture}
\end{minipage}
\end{frame}

\begin{frame}[containsverbatim]{Blocks and threads identification}

\begin{minipage}{5cm}
\begin{itemize}
\item Each thread and block have Ids :
  \begin{itemize}
  \item \textcolor{orange}{Each thread can know on which data he must work}
  \item \textcolor{orange}{Group ID} : 1D, 2D or 3D
  \item \textcolor{orange}{Thread ID}: 1D, 2D or 3D, local or global.
  \end{itemize}
\item The n--dim model simply the memory adressing
  \begin{itemize}
    \item \textcolor{orange}{Image processing}
    \item \textcolor{orange}{Solving EDP on volumes or surfaces}
    \item \ldots
  \end{itemize}
\end{itemize}
\end{minipage}
\begin{minipage}[c]{55mm}
\begin{tikzpicture}{56mm}
\draw[fill=cyan] (15mm,0mm) rectangle (5cm,7cm);
\node at (20mm,68mm){\tiny \textbf{device}};
\draw[fill=green] (17mm,41mm) rectangle (48mm,65mm);
\node at (25mm,62mm){\tiny \textbf{Grid 1}};
\draw[fill=yellow] (22mm,55mm) node[draw,fill] (B00) {\tiny
  \begin{minipage}{6mm}Group\\(0,0)\end{minipage}};
\draw[fill=yellow] (32mm,55mm) node[draw,fill] (B10) {\tiny
  \begin{minipage}{6mm}Group\\(1,0)\end{minipage}};
\draw[fill=yellow] (42mm,55mm) node[draw,fill] (B20) {\tiny
  \begin{minipage}{6mm}Group\\(2,0)\end{minipage}};

\draw[fill=yellow] (22mm,46mm) node[draw,fill] (B01) {\tiny
  \begin{minipage}{6mm}Group\\(0,1)\end{minipage}};
\draw[fill=yellow] (32mm,46mm) node[draw,fill] (B11) {\tiny
  \begin{minipage}{6mm}Group\\(1,1)\end{minipage}};
\draw[fill=yellow] (42mm,46mm) node[draw,fill] (B21) {\tiny
  \begin{minipage}{6mm}Group\\(2,1)\end{minipage}};

\draw[thick,dashed] (27.7mm,42.5mm) -- (6.8mm,2.5mm);
\draw[thick,dashed] (36mm  ,42.5mm) -- (56mm,2.5mm);
\draw[thick,dashed] (27.7mm,49.5mm) -- (6.8mm,32mm);
\draw[thick,dashed] (36mm  ,49.5mm) -- (56mm,32mm);

\draw[fill=yellow, fill opacity=0.7] 
  (6.8mm,2.5mm) rectangle (56mm,32mm);
\node at (14mm,29mm) {\tiny Group (1,1)};
\matrix[fill=red,matrix anchor=south west,nodes=draw] at (13mm,3mm) {
\node{\begin{minipage}{6mm}\tiny Thread (0,0)\end{minipage}};  &&
\node{\begin{minipage}{6mm}\tiny Thread (1,0)\end{minipage}};  && 
\node{\begin{minipage}{6mm}\tiny Thread (2,0)\end{minipage}};  &&
\node{\begin{minipage}{6mm}\tiny Thread (3,0)\end{minipage}};  \\
\node{\begin{minipage}{6mm}\tiny Thread (0,1)\end{minipage}};  && 
\node{\begin{minipage}{6mm}\tiny Thread (1,1)\end{minipage}};  &&
\node{\begin{minipage}{6mm}\tiny Thread (2,1)\end{minipage}};  &&
\node{\begin{minipage}{6mm}\tiny Thread (3,1)\end{minipage}};  \\
\node{\begin{minipage}{6mm}\tiny Thread (0,2)\end{minipage}};  && 
\node{\begin{minipage}{6mm}\tiny Thread (1,2)\end{minipage}};  &&
\node{\begin{minipage}{6mm}\tiny Thread (2,2)\end{minipage}};  &&
\node{\begin{minipage}{6mm}\tiny Thread (3,2)\end{minipage}};  \\ 
};

\end{tikzpicture}
\end{minipage}
\end{frame}


\begin{frame}[containsverbatim]{Intrinsec fonctions for groups and threads}
\begin{minipage}{55mm}
\begin{itemize}
\item \textcolor{blue}{Reserved words for group}
  \begin{itemize}
  \item \textcolor{orange}{\small get\_local\_id({\tiny$\begin{array}{c}0\\1\\2\end{array}$})} :
    position of the thread inside the  group;
  \item \textcolor{orange}{\small get\_local\_size({\tiny$\begin{array}{c}0\\1\\2\end{array}$})}: 
    dimensions of the group.
  \end{itemize}
\item \textcolor{blue}{Reserved words for grids} :
  \begin{itemize}
  \item \textcolor{orange}{\small get\_global\_id({\tiny$\begin{array}{c}0\\1\\2\end{array}$})}
  : position of thread inside the grille;
  \item \textcolor{orange}{\small get\_group\_id({\tiny$\begin{array}{c}0\\1\\2\end{array}$})}
    : position of group inside the grid
  \item \textcolor{orange}{\small get\_num\_groups({\tiny$\begin{array}{c}0\\1\\2\end{array}$})} 
    define the dimension of the grid
  \end{itemize}
\end{itemize}
\end{minipage}
\begin{minipage}[c]{50mm}
\begin{tikzpicture}{56mm}
\draw[fill=yellow] (0mm,70mm) rectangle (50mm,42mm);

\matrix[fill=red,matrix anchor=south west,draw,nodes=draw] at (12mm,49mm) {
\node{\begin{minipage}{6mm}\tiny $T_{000}$\end{minipage}};  &&
\node{\begin{minipage}{6mm}\tiny $T_{100}$\end{minipage}};  && 
\node{\begin{minipage}{6mm}\tiny $T_{200}$\end{minipage}};  &&
\node{\begin{minipage}{6mm}\tiny $T_{300}$\end{minipage}};  \\
\node{\begin{minipage}{6mm}\tiny $T_{010}$\end{minipage}};  && 
\node{\begin{minipage}{6mm}\tiny $T_{110}$\end{minipage}};  &&
\node{\begin{minipage}{6mm}\tiny $T_{210}$\end{minipage}};  &&
\node{\begin{minipage}{6mm}\tiny $T_{310}$\end{minipage}};  \\
\node{\begin{minipage}{6mm}\tiny $T_{020}$\end{minipage}};  && 
\node{\begin{minipage}{6mm}\tiny $T_{120}$\end{minipage}};  &&
\node{\begin{minipage}{6mm}\tiny $T_{220}$\end{minipage}};  &&
\node{\begin{minipage}{6mm}\tiny $T_{320}$\end{minipage}};  \\ 
};


\matrix[fill=red,matrix anchor=south west,draw,nodes=draw] at (6mm,43mm) {
\node{\begin{minipage}{6mm}\tiny $T_{001}$\end{minipage}};  &&
\node{\begin{minipage}{6mm}\tiny $T_{101}$\end{minipage}};  && 
\node{\begin{minipage}{6mm}\tiny $T_{201}$\end{minipage}};  &&
\node{\begin{minipage}{6mm}\tiny $T_{301}$\end{minipage}};  \\
\node{\begin{minipage}{6mm}\tiny $T_{011}$\end{minipage}};  && 
\node{\begin{minipage}{6mm}\tiny $T_{111}$\end{minipage}};  &&
\node{\begin{minipage}{6mm}\tiny $T_{211}$\end{minipage}};  &&
\node{\begin{minipage}{6mm}\tiny $T_{311}$\end{minipage}};  \\
\node{\begin{minipage}{6mm}\tiny $T_{021}$\end{minipage}};  && 
\node{\begin{minipage}{6mm}\tiny $T_{121}$\end{minipage}};  &&
\node{\begin{minipage}{6mm}\tiny $T_{221}$\end{minipage}};  &&
\node{\begin{minipage}{6mm}\tiny $T_{321}$\end{minipage}};  \\ 
};
\draw[<->,thick] (12mm,67mm) -- node[above]{\tiny get\_local\_size(0)} (49.5mm,67mm);
\draw[<->,thick] (4mm,43mm) -- node[above,sloped]{\tiny get\_local\_size(1)} (4mm,60mm);
\draw[<->,thick] (6mm,60mm) -- node[above,sloped]{\tiny get\_local\_size(2)} (12mm,67mm);
\node at (4.5mm,69mm) {\tiny Group 1,1};

\draw[fill=green ] (0mm,0mm)  rectangle (50mm,38mm);
\matrix[fill=yellow,matrix anchor=south west,draw,nodes=draw] at (6mm,10mm) {
\node{\begin{minipage}{6mm}\tiny Group (0,0)\end{minipage}};  &&
\node{\begin{minipage}{6mm}\tiny Group (1,0)\end{minipage}};  && 
\node{\begin{minipage}{6mm}\tiny Group (2,0)\end{minipage}};  &&
\node{\begin{minipage}{6mm}\tiny Group (3,0)\end{minipage}};  \\
\node{\begin{minipage}{6mm}\tiny Group (0,1)\end{minipage}};  && 
\node{\begin{minipage}{6mm}\tiny Group (1,1)\end{minipage}};  &&
\node{\begin{minipage}{6mm}\tiny Group (2,1)\end{minipage}};  &&
\node{\begin{minipage}{6mm}\tiny Group (3,1)\end{minipage}};  \\
};
\draw[<->,thick] (6mm,28mm) -- node[above]{\tiny get\_num\_groups(0)} (43.5mm,28mm);
\draw[<->,thick] (4mm,10mm) -- node[above,sloped]{\tiny get\_num\_groups(1)} (4mm,27mm);
\node at (5mm,35mm) {\tiny Grid 1};
%\draw[step=0.1cm,color=gray!50!white] (0,0) grid (5cm,7cm);
%\draw[step=0.5cm,color=gray] (0,0) grid (5cm,7cm);

\end{tikzpicture}
\end{minipage}
\end{frame}

\section{API for GPU}

\begin{frame}{Property of these APIs : easy to learn and light}
\begin{itemize}
\item The Api is an extension of C language
  \textcolor{green}{$\rightarrow$} easy to learn;

\item Harware is build for light execution and schedule of the threads;
\textcolor{green}{$\rightarrow$} High performance
\end{itemize}
\end{frame}

\begin{frame}[containsverbatim]{Memory allocation}

The Host program (on CPU) must allocate and free memory on GPU
\begin{itemize}
\item \textcolor{blue}{\texttt{cl::Buffer}}
  \begin{itemize}
  \item Allocate objects on \textbf{global memory} of GPU
  \item Two parameters :
    \begin{enumerate}
    \item Size of the allocated object;
    \item Kind of array (read only, copied array, write only, \ldots)
    \end{enumerate}
  \end{itemize}
\item \textcolor{blue}{Destruction of buffer}
  \begin{itemize}
  \item Free object from the global memory of GPGPU;
  \end{itemize}
\end{itemize}

\begin{exampleblock}{Ex.: Allocate an matrix 1024*1024 in simple precision}
    \begin{lstlisting}
#define MATRIX_SIZE 1024*1024
int size = MATRIX_SIZE*sizeof(float);
cl::Buffer MyMatrixOnDevice( context, CL_MEM_WRITE, size) ;
    \end{lstlisting}
\end{exampleblock}
\end{frame}

\begin{frame}{Exchanging data between CPU and GPGPU}

Host program must explicitly copy data from host to device and device to host.

\begin{itemize}
\item Try to limit the number of copies which are slow (go through PCI-Express bus)
\item Copies can be synchronous or asynchronous;
\item Some smart functions are able to copy or not, depending if the global memory
  are different or not for host and device.
\end{itemize}
\end{frame}


\section{Optimisation on GPGPU}

\begin{frame}{Performance of kernels}

\begin{center}\Large
Building a high optimized application on GPU
\end{center}

\begin{itemize}
\item GPU Kernels and threads
\item Warps
\item Memory : Texture and constant memory
\item Shared memory
\item Size of blocks
\end{itemize}
\end{frame}

\subsection{GPU kernels and threads}

\begin{frame}{GPU Kernels and threads}
\begin{itemize}
\item Some parallel part of an application will be executed by GPU as Kernel;
  \begin{itemize}
  \item On old GPU hardware (before Fermi for NVidia), once kernel can be
    executed in the same time on device;
  \item Several threads execute the kernels;
  \end{itemize}
\item Differences between GPU threads and CPU threads :
  \begin{itemize}
  \item GPU threads are very light :
    \begin{itemize}
    \item Create a thread is free (no time spended);
    \item Instantaneous switch between threads.
    \end{itemize}
  \item GPUs must use more than 1000 thread to be efficient.
    \begin{itemize}
    \item CPUs use few threads (4, 8, 16 threads)
    \end{itemize}
  \end{itemize}
\end{itemize}
\end{frame}

\begin{frame}[containsverbatim]{Array of threads}

A GPU Kernel is executed on an n--dimensional array of thread (n=1,2 or 3)
\begin{itemize}
\item All threads execute same code
\item Each thread owns a ID used to compute the memory adresses
and control the conditional jump (if, and so\ldots)
\end{itemize}

\begin{tikzpicture}
\matrix[matrix anchor=south west,nodes=draw] at (0mm,2cm) {
\node[fill=green!90!blue,text height=10pt] (n1) {1}; & 
\node[fill=green!90!blue,text height=10pt] (n2) {2}; & 
\node[fill=green!90!blue,text height=10pt] (n3) {3}; & 
\node[fill=green!90!blue,text height=10pt] (n4) {4}; & 
\node[fill=green!90!blue,text height=10pt] (n5) {5}; & 
\node[fill=green!90!blue,text height=10pt] (n6) {6}; \\
};

\node[anchor=south west] at (0mm,0cm) {
  \begin{lstlisting}
    float x = input[get_global_id(0)];
    float y = func(x);
    output[get_global_id(0)] = y;
  \end{lstlisting}
};

\draw[->,decorate, decoration=snake] (n1.south) -- +(7.5mm,-7.5mm);
\draw[->,decorate, decoration=snake] (n2.south) -- +(7.5mm,-7.5mm);
\draw[->,decorate, decoration=snake] (n3.south) -- +(7.5mm,-7.5mm);
\draw[->,decorate, decoration=snake] (n4.south) -- +(7.5mm,-7.5mm);
\draw[->,decorate, decoration=snake] (n5.south) -- +(7.5mm,-7.5mm);
\draw[->,decorate, decoration=snake] (n6.south) -- +(7.5mm,-7.5mm);

\matrix[matrix anchor=south west,nodes=draw] at (13mm,-13mm) {
\node[fill=orange!90!blue,text height=10pt] (m1) {1}; & 
\node[fill=orange!90!blue,text height=10pt] (m2) {2}; & 
\node[fill=orange!90!blue,text height=10pt] (m3) {3}; & 
\node[fill=orange!90!blue,text height=10pt] (m4) {4}; & 
\node[fill=orange!90!blue,text height=10pt] (m5) {5}; & 
\node[fill=orange!90!blue,text height=10pt] (m6) {6}; \\
};

\draw[<-,decorate, decoration=snake] (m1.north) -- +(-7.5mm,7.5mm);
\draw[<-,decorate, decoration=snake] (m2.north) -- +(-7.5mm,7.5mm);
\draw[<-,decorate, decoration=snake] (m3.north) -- +(-7.5mm,7.5mm);
\draw[<-,decorate, decoration=snake] (m4.north) -- +(-7.5mm,7.5mm);
\draw[<-,decorate, decoration=snake] (m5.north) -- +(-7.5mm,7.5mm);
\draw[<-,decorate, decoration=snake] (m6.north) -- +(-7.5mm,7.5mm);

\end{tikzpicture}
\end{frame}

\begin{frame}[containsverbatim]{Thread ID}

The local ID of a thread in a group is :
\begin{lstlisting}
t_id = get_local_id(0) + 
       get_local_id(1)*get_local_size(0)
       get_local_id(2)*(get_local_size(0)*get_local_size(1));
\end{lstlisting}

\begin{itemize}
\item \textcolor{orange}{\texttt{get\_local\_id({\tiny$\begin{array}{c}0\\1\\2\end{array}$})}} : 
  Indice of thread in direction 0, 1 or 2
\item \textcolor{orange}{\texttt{get\_local\_size({\tiny$\begin{array}{c}0\\1\\2\end{array}$})}} :
  Size of groupe in dimension 0, 1 or 2
\end{itemize}
\end{frame}

\begin{frame}[containsverbatim]{Thread ID(2)}

The global ID of a thread in a grid is :
\begin{lstlisting}
t_id = get_global_id(0) + 
       get_global_id(1)*get_global_size(0)
       get_global_id(2)*(get_global_size(0)*get_global_size(1));
\end{lstlisting}

It's possible to compute global indice from local indice :
\begin{lstlisting}
t_id = get_local_id(0) + get_group_id(0)*get_local_size(0) + 
       get_local_id(1)*get_global_size(0) + 
       get_group_id(1)*get_local_size(1) +
       get_local_id(2)*(get_global_size(0)*get_global_size(1)) +
       get_group_id(2)*get_local_size(2);
\end{lstlisting}
\end{frame}

\begin{frame}[containsverbatim]{Example : Basic 2D Addition in a group}

Kernel for a 2D arrays addition :
{\scriptsize
\begin{lstlisting}
kernel void 
addMatrix(global float* A, global float* B, global float* C, int N) {
 int ind=get_global_id(0) + get_global_id(1)*get_global_size(0);
 if ((get_global_id(0)<N)&&(get_global_id(1)<N)) C[ind]=A[ind]+B[ind];
}
\end{lstlisting}
}
API calling the previous kernel :
{\scriptsize
\begin{lstlisting}
void addMat(float* A, float* B, float* C, int N) {
 cl::Buffer Ad(context, CL_MEM_READ_ONLY|CL_MEM_USE_HOST_PTR, 
               N*N*sizeof(float), Ah);
 cl::Buffer Bd(context, CL_MEM_READ_ONLY|CL_MEM_USE_HOST_PTR, 
               N*N*sizeof(float), Bh);
 cl::Buffer Cd(context, CL_MEM_READ_WRITE|CL_MEM_COPY_HOST_PTR, 
               N*N*sizeof(float), Ch);
 // Calcule la configuration d'execution du noyau
 int grpSz = 16, grdSz=(N%grpSz>0 ? grpSz*(N/grpSz+1),N);
 addMatKer.setArg(0,Ad); addMatKer.setArg(1,Bd);
 addMatKer.setArg(2,Cd); addMatKer.setArg(3,N);
 q.enqueueNDRangeKernel(addMatKer, cl::NullRange,
                 cl::NDRange(grdSz,grdSz),cl::NDRange(grpSz,grpSz));
 q.finish();
 // Copie le resultat sur le CPU
 q.enqueueReadBuffer(Ad, true, 0, N*N*sizeof(float), Ah);
}
\end{lstlisting}
}
\end{frame}

\subsection{Warps}

\begin{frame}{Warps}

On GPU :
\begin{itemize}
\item Each group of threads is splitted in warps
\item Each warp contains same number of threads (32 on NVIDIA)
\item Each warp contains consecutive threads, first warp containing
  thread number 0;
\item Each warp is running by the multiprocessor as pure SIMD device
\item Divergent branchs inside a same warp are serialized :
  \begin{itemize}
  \item If all threads into a warp go to the same branch, no time penality
  \item If some threads into a warp go to another branch, the warp spend
    the addition of time for both branch;
  \item If all threads go to $n$ branchs, the warp spend the addition of 
    time of $n$ branchs;
  \end{itemize}
\end{itemize}
\end{frame}

\subsection{Mémoire : texture et mémoire constante}

\begin{frame}{Texture and constant memory}

\begin{itemize}
\item Texture memory contains a cache memory
  \begin{itemize}
  \item Texture memory can be seen as  1D, 2D or 3D array
  \item In read only \alert{but}
  \item An 1D texture can be associated to the global memory
    \begin{itemize}
    \item Using cache memory to read
    \item Write in global memory is possible
    \end{itemize}
  \end{itemize}
\item Constant memory use cache memory
  \begin{itemize}
  \item 4 cycles to read in a simple half-warp
    \begin{itemize}
    \item Total cost is 4 cycles if all threads into a warp read same address
    \item Total cost is 64 cycles if all threads into a warp read different addresses
    \end{itemize}
  \end{itemize}
\end{itemize}
\end{frame}

\subsection{Shared memory}

\begin{frame}{Shared memory}

\begin{itemize}
\item 100 times faster than global memory
\item Threads of same group can cooperate thanks to the shared memory
  \begin{itemize}
  \item Max 48 KBytes per multiprocessors on NVIDIA (beware, shared with the cache memory)
  \end{itemize}
\end{itemize}
\end{frame}

\begin{frame}{Shared memory ; performance issue}

\begin{itemize}
\item Ideal case :
  \begin{itemize}
  \item All threads of a warp access globally to a different consecutive addresses
  \item All threads of a warp access to the same variable
  \end{itemize}
\item Worst case :
  \begin{itemize}
  \item Some threads read the same variable (but not all threads)
  \item Memory cost = max \# of access of the same variable
  \end{itemize}
\end{itemize}
\end{frame}

\begin{frame}{Ideal case}

\begin{minipage}[c]{3cm}
\begin{tikzpicture}
\draw[xstep=1cm,ystep=0.5cm] (0,0) grid (1cm,8cm);
\draw[xstep=1cm,ystep=0.5cm] (1.99cm,0) grid (3cm,8cm);
\foreach \b / \x in {0 / 0.25,1 / 0.75,2 / 1.25, 3/1.75, 4/2.25,
  5/2.75, 6/3.25, 7/3.75, 8/4.25, 9/4.75, 10/5.25, 11/5.75,
  12/6.25, 13/6.75, 14/7.25,15 / 7.75} {
\draw (0.5cm,\x cm) node {\tiny Thread \b};
\draw (2.5cm,\x cm) node {\tiny Bank   \b};
\draw[thick, color=blue,->] (1cm, \x cm) -- (2cm, \x cm);
}
\end{tikzpicture}
\end{minipage}
\begin{minipage}[c]{4cm}
\small
\textbf{Ideal access}
Each thread of the warp access to different successive variables
\end{minipage}
\begin{minipage}[c]{3cm}
\begin{tikzpicture}
\draw[xstep=1cm,ystep=0.5cm] (0,0) grid (1cm,8cm);
\draw[xstep=1cm,ystep=0.5cm] (1.99cm,0) grid (3cm,8cm);
\foreach \b / \x in {0 / 0.25,1 / 0.75,2 / 1.25, 3/1.75, 4/2.25,
  5/2.75, 6/3.25, 7/3.75, 8/4.25, 9/4.75, 10/5.25, 11/5.75,
  12/6.25, 13/6.75, 14/7.25,15 / 7.75} {
\draw (0.5cm,\x cm) node {\tiny Thread \b};
\draw (2.5cm,\x cm) node {\tiny Bank   \b};
}
\foreach \b / \x in {0.25 / 1.25,0.75 / 0.25,1.25 / 2.75, 1.75/1.75, 2.25/2.25,
  2.75/0.75, 3.25/3.25, 3.75/3.75, 4.25/5.75, 4.75/4.75, 5.25/5.25, 5.75/4.25,
  6.25/6.25, 6.75/6.75, 7.25/7.25,7.75 / 7.75} {
\draw[thick, color=blue,->] (1cm, \b cm) -- (2cm, \x cm);
}
\end{tikzpicture}
\end{minipage}
\end{frame}

\begin{frame}{Ideal and worst case}

\begin{minipage}[c]{3cm}
\begin{tikzpicture}
\draw[xstep=1cm,ystep=0.5cm] (0,0) grid (1cm,8cm);
\draw[xstep=1cm,ystep=0.5cm] (1.99cm,0) grid (3cm,8cm);
\foreach \b / \x in {0 / 0.25,1 / 0.75,2 / 1.25, 3/1.75, 4/2.25,
  5/2.75, 6/3.25, 7/3.75, 8/4.25, 9/4.75, 10/5.25, 11/5.75,
  12/6.25, 13/6.75, 14/7.25,15 / 7.75} {
\draw (0.5cm,\x cm) node {\tiny Thread \b};
\draw (2.5cm,\x cm) node {\tiny Bank   \b};
\draw[thick, color=blue,->] (1cm, \x cm) -- (2cm, 4.75 cm);
}
\end{tikzpicture}
\end{minipage}
\begin{minipage}[c]{4cm}
\small
\begin{itemize}
\item[$\leftarrow$] Each thread read the same variable
  (broadcasting)
\item[] Several threads access to the same variable :\alert{conflicts $\rightarrow$}
\end{itemize}
\end{minipage}
\begin{minipage}[c]{3cm}
\begin{tikzpicture}
\draw[xstep=1cm,ystep=0.5cm] (0,0) grid (1cm,8cm);
\draw[xstep=1cm,ystep=0.5cm] (1.99cm,0) grid (3cm,8cm);
\foreach \b / \x in {0 / 0.25,1 / 0.75,2 / 1.25, 3/1.75, 4/2.25,
  5/2.75, 6/3.25, 7/3.75, 8/4.25, 9/4.75, 10/5.25, 11/5.75,
  12/6.25, 13/6.75, 14/7.25,15 / 7.75} {
\draw (0.5cm,\x cm) node {\tiny Thread \b};
\draw (2.5cm,\x cm) node {\tiny Bank   \b};
}
\foreach \b / \x in {0.25 / 1.25,0.75 / 0.25,1.25 / 1.75, 1.75/1.75, 2.25/2.25,
  2.75/0.75, 3.25/3.25, 3.75/3.25, 4.25/3.25, 4.75/4.75, 5.25/5.25, 5.75/4.25,
  6.25/6.25, 6.75/6.75, 7.25/7.25,7.75 / 7.75} {
\draw[thick, color=red,->] (1cm, \b cm) -- (2cm, \x cm);
}
\end{tikzpicture}
\end{minipage}
\end{frame}

\subsection{Conditions branch and unrolling}

\begin{frame}{Conditions branch and unrolling}

  \begin{itemize}
  \item Avoid conditional branches inside the kernel
    \begin{itemize}
    \item Minimize divergent branches to avoid time penality
    \item \underline{Tricks} :
      \begin{itemize}
      \item Try to specialize the kernel in function of the condition
        (size by example)
      \item Try to use mask technics
      \end{itemize}
    \end{itemize}
  \item  Avoid loops inside the kernel
    \begin{itemize}
    \item Try to unroll a loop at hand if the length of the loop is known
    \end{itemize}
\end{itemize}
\end{frame}

\subsection{Size of groups}

\begin{frame}{Optimal number of threads per group}

  \begin{itemize}
  \item Choose the number of threads per group as a multiple of the size of a warp;
    \begin{itemize}
    \item Try to avoid unused threads in a warp
    \end{itemize}
  \item Lot of threads per group = better memory access optimization
    \begin{itemize}
    \item But calling a kernel with lof of threads can crash the kernel if too much
      registers are used;
    \end{itemize}
  \item Heuristics:
    \begin{itemize}
    \item Minimal threads for the hardware : 64 threads per group
      \begin{itemize}
      \item Only if lot of concurrent blocks
      \end{itemize}
    \item 192 or 256 threads are better choices :
      \begin{itemize}
      \item Generally enough registers to compile and executing kernels;
      \end{itemize}
    \item Depending of your kernel computation : let's try !
    \end{itemize}
  \end{itemize}
\end{frame}

\begin{frame}{Heuristics size of grid/group}

  \begin{itemize}
  \item \# of groups $>$ \# of multiprocessors
    \begin{itemize}
    \item Each multiprocessor has a group to execute
    \end{itemize}
  \item \# of groups / \# of multiprocessors $>$ 2
    \begin{itemize}
    \item Several groups can be in concurrence in a multiprocessor
    \item Groups which don't wait with a \texttt{barrier} are always active
    \item Relativ to available ressources -- registers, shared memory
    \end{itemize}
  \item \# of groups $>$ 100 to be efficient on futurs hardware
    \begin{itemize}
    \item Groups executed in pipeline on multiprocessors
    \item 1000 groups per grid should be efficient for future generation of GPUs
    \end{itemize}
  \end{itemize}
\end{frame}

\end{document}
