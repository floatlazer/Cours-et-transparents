\documentclass[fleqn,11pt]{article}
\usepackage[top=3cm,bottom=3cm,left=3cm,right=3cm,headsep=10pt,a4paper]{geometry} % Page margins

\usepackage{graphicx} % Required for including pictures
\graphicspath{{Pictures/}} % Specifies the directory where pictures are stored

\usepackage{tikz} % Required for drawing custom shapes
\usepackage{dsfont}
\usepackage{enumitem} % Customize lists
\setlist{nolistsep} % Reduce spacing between bullet points and numbered lists

\usepackage{booktabs} % Required for nicer horizontal rules in tables
\usepackage{xcolor} % Required for specifying colors by name
\definecolor{ocre}{RGB}{243,102,25} % Define the orange color used for highlighting throughout the book
\usepackage{microtype} % Slightly tweak font spacing for aesthetics
%\usepackage[utf8]{inputenc} % Required for including letters with accents
%\usepackage[T1]{fontenc} % Use 8-bit encoding that has 256 glyphs

\usepackage{calc} % For simpler calculation - used for spacing the index letter headings correctly
\usepackage{makeidx} % Required to make an index
\makeindex % Tells LaTeX to create the files required for indexing
\usepackage[many]{tcolorbox}
\usepackage{listings}
\usepackage{smartdiagram}
\usetikzlibrary{shadows, arrows, decorations.pathmorphing, fadings, shapes.arrows, positioning, calc, shapes, fit, matrix}
\usepackage{polyglossia}
\usepackage{caption}
\usepackage{subcaption}

\definecolor{lightblue}{RGB}{0,200,255} 
\definecolor{paper}{RGB}{239,227,157}

\pgfdeclarelayer{background}
\pgfdeclarelayer{foreground}
\pgfsetlayers{background,main,foreground}

\lstset{%
  basicstyle=\footnotesize,
  frame=single,
  keywordstyle=\color{blue},
  language=bash,
  commentstyle=\color{red},
  stringstyle=\color{brown},
  keepspaces=true,
  showspaces=false,
  tabsize=2
}

\title{Git: Résumé des commandes}
\author{Juvigny Xavier}
\date{Mars 2017}

\newtheorem{prop}{Propriétés }
\newtheorem{remark}{Remarque }

\begin{document}
\maketitle

\section{Configuration}

\begin{itemize}
\item Nom de l'utlisateur 
\begin{lstlisting}
git config --global user.name "Olivier.Martin"
\end{lstlisting}
\item E-mail de l'utilisateur
\begin{lstlisting}
git config --global user.email "Olivier.Martin@ensta.fr"
\end{lstlisting}
\item Choix de l'éditeur utilisé pour git
\begin{lstlisting}
git config --global core.editor subl
\end{lstlisting}
\item Choix de l'outil de comparaison de deux fichiers :
\begin{lstlisting}
git config --global merge.tool diff
\end{lstlisting}
\item Affiche la configuration choisie pour git :
\begin{lstlisting}
git config --list
\end{lstlisting}
\end{itemize}

\section{Commandes principales}

\begin{itemize}
\item \'Etat des fichiers (modifiés, effacés, soumis, \ldots ) : 
\begin{lstlisting}
git status
\end{lstlisting}
\item Affichage des diverses branches : 
\begin{lstlisting}
git branch
\end{lstlisting}
\item Créer une nouvelle branche :
\begin{lstlisting}
git branch nom_de_ma_branche
\end{lstlisting}
\item Changer de branche :
\begin{lstlisting}
git checkout nom_de_ma_branch
\end{lstlisting}
\item Première soumission :
\begin{lstlisting}
git add .
git commit - m "initial commit"
\end{lstlisting}
\item Soumissions suivantes :
\begin{lstlisting}
git add chemin_vers_mon_fichier
git commit -m "message du commit"
\end{lstlisting}
\item Annule la dernière soumission et les modifications associées :
\begin{lstlisting}
git reset --hard md5_commit
git push --force
\end{lstlisting}
\item Mettre à jour le dépot local :
\begin{lstlisting}
git pull
\end{lstlisting}
\item Envoyer sa soumission au dépot distant :
\begin{lstlisting}
git push
\end{lstlisting}
\item Supprimer un fichier du répertoire et de l'index des fichiers
gérés par git :
\begin{lstlisting}
git rm nom_du_fichier
\end{lstlisting}
\item Supprimer un fichier uniquement de l'index des fichiers gérés par git :
\begin{lstlisting}
git rmg --cached nom_du_fichier
\end{lstlisting}
\end{itemize}

\section{Afficher les différences entre diverses versions}

\begin{itemize}
\item Affiche la différence entre le contenu du dernier commit et celui du répertoire de travail. Cela correspond à ce qui serait commité par 
\lstinline@git commit -a@.
\begin{lstlisting}
git diff HEAD
\end{lstlisting}
\item Affiche la différence entre le contenu pointé par A et celui pointé par B.
\begin{lstlisting}
git diff A B
\end{lstlisting}
\item Diff entre un dossier présent sur deux branches
\begin{lstlisting}
git diff master..MA_BRANCH chemin/vers/mon_dossier
\end{lstlisting}
\end{itemize}

\section{Afficher les dernières soumissions}

\begin{itemize}
\item Afficher les dernières soumissions :
\begin{lstlisting}
git log
\end{lstlisting}
\item Affiche les $X$ dernières soumissions  :
\begin{lstlisting}
git log -n X
\end{lstlisting}
\item Affiche un ensemble de commits par date :
\begin{lstlisting}
git log --since=date --until=date
\end{lstlisting}
\item Représentation de l'historique à partir de \texttt{HEAD} ( soumission/branche ) :
\begin{lstlisting}
git log --oneline --graph --decorate
\end{lstlisting}
\item Représentation de l’historique à partir d'un fichier (commit / branch) :
\begin{lstlisting}
git log --oneline --graph --decorate nom_du_fichier
\end{lstlisting}
\end{itemize}

\section{Annuler des commits}

\subsection{Version soft}

Seul le commit est retiré de Git : 
vos fichiers, eux, restent modifiés. Vous pouvez alors à nouveau changer vos fichiers si besoin est et refaire un commit.

\begin{lstlisting}
git reset HEAD^
\end{lstlisting}

Pour indiquer à quel commit on souhaite revenir, il existe plusieurs notations :
\begin{itemize}
\item \verb@HEAD@ : dernier commit ;
\item \verb@HEAD^@ : avant-dernier commit ;
\item \verb@HEAD^^@ : avant-avant-dernier commit ;
\item \verb@HEAD~2@ : avant-avant-dernier commit (notation équivalente) ;
\item \verb@d6d98923868578a7f38dea79833b56d0326fcba1@ : indique un numéro de commit précis;
\end{itemize}

\subsection{Version hard}

Si vous voulez annuler votre dernier commit et les changements effectués dans les fichiers, 
il faut faire un reset hard. Cela annulera sans confirmation tout votre travail !

\begin{itemize}
\item Annuler les commits et perdre tous les changements

\begin{lstlisting}
git reset --hard HEAD^
\end{lstlisting}

\item Annuler les modifications d’un fichier avant un commit

Si vous avez modifié plusieurs fichiers mais que vous n’avez pas encore envoyé le commit et que vous voulez restaurer un fichier tel qu’il était au dernier commit :

\begin{lstlisting}
git checkout nom_du_fichier
\end{lstlisting}

\item Annuler/Supprimer un fichier avant un commit

Supposer que vous venez d’ajouter un fichier à Git avec git add et que vous vous apprêtez à le "commiter". 
Cependant, vous vous rendez compte que ce fichier est une mauvaise idée et vous voulez annuler votre git add.

Il est possible de retirer un fichier qui avait été ajouté pour être « commité » en procédant comme suit :

\begin{lstlisting}
git reset HEAD -- nom_du_fichier_a_supprimer
\end{lstlisting}
\end{itemize}

\section{Travailler avec Github}

Récuperer le repo sur Github :

\begin{lstlisting}
git clone https://github.com/JuvignyEnsta/Projet
\end{lstlisting}

Mon repo est composé d'au moins deux branches.

\begin{itemize}
\item[\textbf{develop}] : dédié au développement et résolution de bug. 
\item[\textbf{master}] : reflète le code en production. Personne ne doit travailler directement sur cette branche.
\end{itemize}

Pour récupérer votre branche develop

\begin{lstlisting}
git branch -a
git checkout origin/develop
git checkout -b develop origin/develop
git branch
\end{lstlisting}

\noindent\textbf{Développement} : Branche develop\\
\textbf{Production} : Branche Master

Pour développer sur la branche develop puis soumettre sur la branche master :

\begin{lstlisting}
# On se met sur la branche master
git checkout master
# On merge la branche develop
git merge develop
# Pour pousser les changements
git push origin master
# Penser à revenir sur develop
git checkout develop
\end{lstlisting}

\end{document}
